Straight away communicate the idea that core exploration should be easier and more approachable and give
and drill it home with a small but daunting core snippet.
 
Haskell is a language explicit about side effects which allows for pretty drastic semantic preserving transformations,
inviting programmers to hand over a lot of optimisation responsibilities. To reasure success one needs to explore
interactions with the compiler.

Generally we want our programs to not only be correct but also terminate in reasonable amount of time while
not consuming a overly large chunk of resources. This collection of additional constraints are examples of 
Non-functional requirements. In trying to meet these requirements, we expect an optimising compiler
to do some of the heavy lifting. Compositional style programming like in Haskell puts even more
faith in the compiler in this regard.
