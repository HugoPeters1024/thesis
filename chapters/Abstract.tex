GHC -- the Haskell compiler -- uses a bespoke intermediate language upon which a number of separate optimisation transformations take place.
From the compositional style of programming, that makes languages like Haskell so attractive, follows that optimisation is essential as to not produce
unreasonably slow binaries.
While generally successful, it is needed at times to inspect this intermediate representation throughout the transformations to understand
why performance is unexpectedly disappointing or has regressed. This has historically been a task reserved for the more hardened and experienced
Haskell developer, and is often done in a primitive manner.
\\

Recent research has explored the ability to include assertions about optimisation that are expected to take place in traditional test suites.
After all we generally want our programs to not only be correct, but also terminate in a reasonable amount of time.
This is an exciting idea, but it does not address the need to inspect the intermediate representation itself and the skill required to do so.
\\

We believe that Core inspection can be streamlined with an interactive tool that allows users to explore and comprehend such
intermediate programs more pleasantly and efficiently. We describe what such a tool may look and how we implemented it. Then we
empirically evaluate our tool by reproducing a real world performance regression in the popular \mono{text} library and show how our tool
could have been of assistance in that situation. Furthermore, we discuss how we used our tool to discover a performance bug in the fusion  
system of contemporary GHC itself.
