\chapter{Background}

\section{A Core Language}
\label{section:background:core_lang}

GHC recognises three distinct stages of the compilation process \cite{haskell_optimisations_1997}.

\begin{enumerate}
  \item \textbf{Frontend}, parsing and type checking, enables to give clear errors that reference the unaltered source code.
  \item \textbf{Middle}, a number of optimisation transformations.
  \item \textbf{Backend}, generates and compiles to machine code, either via C-- or LLVM.
\end{enumerate}

The middle section is tasked with doing all the optimisation work it can, leaving only those optimisations to the backend
it can not otherwise perform. Within the middle section, the work is further split into substages, where each transformation
is a separate, composable pass. An obvious benefit to this approach is that each pass can be tested independently. Moreover, because
the types are preserved throughout the middle section, it can be verified that each transformation preserves type correctness;
Providing some evidence towards the correctness of the transformation.

Regardless, the Haskell source language itself is not a good target for optimisation. The source has to be
rich and expressive, which yields an AST datatype of over 100 constructors. Any transformation over such a datatype has far too
many cases to be considered practical to implement; Not to mention the myriad of bug-inducing edge cases. To overcome this issue, the
middle section first performs a desugaring pass, translating the source language into a far simpler intermediate language
called \textit{Core}. 

Core --- which is how we will refer to GHC intermediate representation going forward --- is designed to be as simple as possible
without being impractical. A testament to that property is the fact that we can fit the entire definition on the page:

\begin{listing}[H]
\begin{minted}{haskell}
data Expr
  | Var Id
  | Lit Literal
  | App Expr Expr
  | Lam Bndr Expr
  | Let Bind Expr
  | Case Expr Bndr Type [Alt]
  | Cast Expr CoercionR -- safe coercions are required for GADTs
  | Tick Tickish Expr  -- annotation that can contain some meta data
  | Type Type
  | Coercion Coercion
  
data Alt = Alt AltCon [Bndr] Expr
 
data Bind
  = NonRec Bndr Expr
  = Rec [(Bndr, Expr)]
  
-- the Var type contains information about
-- a variable, like it's name, a unique identifier
-- and analysis results
type Bndr = Id
type Id = Var

type Program = [Bind]
\end{minted}
\label{code:core_def}
\caption{An ever so slightly simplified version of the Core Language}
\end{listing}

Writing an optimisation transformation essentially of type \hs{Program -> Program} does not now seem
as daunting, given the drastically reduced number of cases to consider. 
Likewise, maintaining and debugging transformations is much less of a strenuous task.

\section{The core2core transformations}

Over its numerous decades of development, the core2core pipeline has been fitted with a myriad of transformations.
It would be impractical to discuss all of them here. However, we will discuss in depth the role of the multiple simplifier
passes, as well as the worker/wrapper transformation. The first because it gives context to the rewrite rules and the latter
because it is a natural bridge to unboxed types; both concepts which will become relevant in discussing the results of this thesis.
Lastly, we zoom in on the analysis results stored in the \mono{Var} type.

\subsection{The simplifier: its parts}

The simplifier is quite literally an indispensable part of the transformation pipeline. Although the parts of the pipeline
are meant to be separable entities, they heavily rely on the simplifier to deal with the cleanup. As such, it has earned
itself the reputation of being the workhorse of the pipeline \cite{ghc_wiki_core2core}.

If you were looking for an atom, you have not found it. The simplifier is in itself again a collection of smaller transformations,
albeit applied repeatedly and interleaved such that they cannot be easily untangled. Each simplifier subpart is a local transformation,
that is, it only looks at the immediate surroundings of the expression. We give a near comprehensive list of each subpart along with an example: 
\cite{ghc_wiki_core2core}

\subsubsection{Constant Folding}
Evaluate expressions that require no runtime information. This is a very logical transformation that does not
require any further considerations.

\begin{listing}[H]
\begin{minted}{haskell}
-- before
3 + 4
-- after
7
\end{minted}
\end{listing}

\subsubsection{Inlining}
\label{section:background:inlining}

Inlining function calls is a well known performance enhancing operation in all language, but especially so in functional ones.
The difference is staggering with a around 10-15\% performance increase for conventional languages, as opposed to
30-40\% for functional languages \cite{haskell_optimisations_1997}.

Unlike constant folding, it is not an ad-hoc no-brainer. An easy case would be to inline a function that is
used exactly once. All that this does is remove the veil that hides the function body for further optimisations:

\begin{listing}[H]
\begin{minted}{haskell}
-- before
f x = x + 1
f 3
-- after
3 + 1
\end{minted}
\end{listing}

However, if the function is used in multiple locations with different arguments, you risk increasing the size of the program
because the function body will be duplicated at each call-site. This is a trade-off that --- although often worth it --- must be
considered on a case to case basis. GHC has a number of heuristics to determine whether inlining is worth it or not. \cite{secrets_of_the_inliner}
Obviously, this cannot be a perfect solution and one can imagine how a small change in the code can suddenly cause the inliner to reverse
its decision; a testament to the non-continuous nature of the compiler.

Besides inlining function calls, Haskell's let bindings also form an opportunity for inlining.
After all, let bindings are often described as syntax sugar for lambda abstractions, but there is an important difference to be considered.
Because let bindings in Haskell are lazily evaluated, it may lead to explosion if the let bound variable is used in a lambda
expression. For example:

\begin{listing}[H]
\begin{minted}{haskell}
-- before, 'expensive' is called at most once
let v = expensive 42
    l = \x -> ... v ...
in map l xs

-- after, 'expensive' is called for each element of 'xs'
let l = \x -> ... expensive 42 ...
in map l xs
\end{minted}
\end{listing}

In this case, inlining would be disastrous for performance and GHC will take great care to avoid it. 

\subsubsection{Case of known constructor}

The case destruction is only the way to get to the WHNF of an expression. This means that inside of a case expression
we have learned information about the variable under scrutiny and can use it to infer the result of outer case expressions.
Consider the following scenario:

\begin{listing}[H]
\begin{minted}{haskell}
safe_tail :: [a] -> [a]
safe_tail ls = case ls of
    [] -> []
    x:xs -> tail ls
\end{minted}
\end{listing}

Which inlining will transform into:

\begin{listing}[H]
\begin{minted}{haskell}
safe_tail :: [a] -> [a]
safe_tail ls = case ls of
    [] -> []
    x:xs -> case ls of
        [] -> error "tail of empty list"
        (x:xs) -> xs
\end{minted}
\end{listing}

Since we scrutinize \mono{ls} again in the inner case, we actually already know that the second case is only ever called,
thus we can eliminate the case expression as a whole.

\begin{listing}[H]
\begin{minted}{haskell}
safe_tail :: [a] -> [a]
safe_tail ls = case ls of
    []   -> []
    x:xs -> xs
\end{minted}
\end{listing}

The removal of the call \mono{error} during this process is a testament to this function being deserving of the \mono{safe} prefix.

\subsubsection{Case of case}

There are cases (no pun intended) where the case-of-know-constructor cannot quite do its job, although it is obvious
that benefits are still to be gained. Consider for example what happens when instead of scrutinizing the same variable,
the outer case scrutinizes the result of the innner case:


\begin{listing}[H]
\begin{minted}{haskell}
-- before (possibly desugared from 'if (not x) then E1 else E2'
-- after also inlining 'not')
case (case x of {True -> False; False -> True}) of
    True -> E1
    False -> E2
\end{minted}
\end{listing}

We might hope to gain something from the information that the inner case has produced by moving the outer case expression
to each branch of the inner one:

\begin{listing}[H]
\begin{minted}{haskell}
case x of
    True -> case False of {True -> E1; False -> E2}
    False -> case True of {True -> E1; False -> E2}
\end{minted}
\end{listing}

Now we can rely on constant folding to simplify all the way down to:

\begin{listing}[H]
\begin{minted}{haskell}
case x of
    True -> E2
    False -> E1
\end{minted}
\end{listing}

Note that we do risk duplicating \mono{E1} and \mono{E2} if the expression does not reduce so completely.
A solution to this are \textit{join points} \cite{compiling_wo_continuations},
which we will not go into here for the sake of brevity.

\subsubsection{Rewrite rules}

We already discussed in \cref{section:introduction:rewrite_rules} how rewrite rules are way to express domain
specific knowledge to the compiler that it otherwise can not practically infer. Applying given rewrite rules is task
also given to the simplifier. It should now be clear that this process can get a little messy since the simplifier
is under no obligation to apply the rules, nor its other tasks, in any particular order. We will get into this issue
more in the next section where we discuss the simplifier at large.

It should be noted that the use of rewrite rules are very common during the compilation of most any Haskell program.
This is because the internal fusion system on list operations are implemented as built in rewrite rules.
We discuss this system more in depth in \cref{section:background:fusion}.



\subsection{The simplifier: its sum}

An attentive reader might have noticed that when explaining one part of the simplifier, we often relied on another to
its job. This begs the question: how does the simplifier determine in which order to run its subparts. The answer to that is
that it does not. The analogy here is that a compiler is a gun and a program is a target. Every program is very different
and we cannot know in advance how to best hit it. Therefore, we must ensure that the compiler --- or this case the simplifier stage ---
has a lot of bullets in its gun to ensure being able to effectively hit a lot of targets \cite{haskell_optimisations_1997}.

A secondary problem with this approach is that one simplifier part may enable another to its job after. Running the simplifier
once is therefore not satisfactory, and we must run it until it reaches a sort of \textit{fixed point}; At least up until some
arbitrary limit since there is no guarantee such a fixed point exists and even if it does that we find it and not get stuck in a loop.

Aforementioned looping behavior can actually occur quite easily due to certain rewrite rules. It is not always the case that
the RHS of a rewrite rules objectively better than the right. It might be that the rewrite is benificial because it \textbf{may}
enable other optimisation to take place afterwards. However, if for some reason that did not turn out to be possible, we may want
to apply the reverse rewrite rule later. This is obviously problematic as we can be ping-pong between rewrite rules forever.
To overcome that issue one can enable/disable rewrites rules in certain phases of simplifier. To understand this we must first
understand when the simplifier is run.

In order, the simplifier is --- rather unintuitively --- interspersed between other transformations as follows:

\begin{enumerate}
  \item \textbf{Gentle} (disables case-of-case transformations)
  \item \textbf{Phase 2}
  \item \textbf{Phase 1}
  \item \textbf{Phase 0}
  \item \textbf{Final}
\end{enumerate}

By default, rewrite rules as well as inlinings can occur in each of these phases but the programmer does have the ability
to specify deviations from this behavior. For example, in the \mono{text} library, we find in the \mono{Data.Text} module
the following snippet:

\begin{listing}[H]
\begin{minted}{haskell}
-- This function gives the same answer as comparing against the result
-- of 'length', but can short circuit if the count of characters is
-- greater than the number, and hence be more efficient.
compareLength :: Text -> Int -> Ordering
compareLength t c = S.compareLengthI (stream t) c
{-# INLINE [1] compareLength #-}

{-# RULES
"TEXT compareN/length -> compareLength" [~1] forall t n.
    compare (length t) n = compareLength t n
#-}
\end{minted}
\end{listing}

Here phase control is used to indicate that \mono{compareLength} should \textbf{only} be inlined in phase 1 and conversely
that the rewrite rule \textbf{compareN/length} may be applied \textbf{except} in phase 1. What this ensures is that the inliner
will not operate on the result of the rewrite rule directly in the same phase. This is supposedly because we expect \mono{compareLength}
to occur in the LHS of a different rewrite rule which is to be desired over inlining at first.

\subsection{The worker/wrapper transformation}
\label{section:background:worker_wrapper}

The \textit{worker/wrapper} transformation is able to change the type of computation (typically a loop) into a \textit{worker} function
along with a \textit{wrapper} that can convert back to the original type. The example listed on the GHC wiki is that of the reverse
function on lists \cite{ghc_wiki_wwrapper}. One might concisely define with direct recursion and the \mono{++} operator:

\begin{listing}[H]
\begin{minted}[linenos]{haskell}
reverse :: [a] -> [a]
reverse [] = []
reverse (x:xs) = reverse xs ++ [x]
\end{minted}
\end{listing}

Of course this is not very efficient as the \mono{++} operator itself is already linear, make the reverse function quadratic; If we
create an auxiliary \textbf{worker} function \mono{reverse'} that takes an extra accumulator argument, we can create a linear version:

\begin{listing}[H]
\begin{minted}[linenos]{haskell}
reverse :: [a] -> [a]
reverse = reverse' []
  where reverse' acc [] = acc
        reverse' acc (x:xs) = reverse' (x:acc) xs
\end{minted}
\end{listing}

Here the \textbf{wrapper} is simply applying the empty list as a starting argument to the function. Thus, by introducing some state
that exists only during the lifetime of the computation, the function has become asymptotically more efficient.

Impressively, the transformation may also greatly improve the constant factor of the runtime by leveraging unboxed types.
Unlike strict languages, the \mono{Int} type in Haskell --- while always representing a 64 bit integer --- is not statically sized.
After all, any value is lazy and may therefore still be unevaluated thunk whose size is unknown at compile time. As a result, \mono{Int}s
are always stored on the heap and thus require runtime allocation. However, \mono{Int} has a strict counterpart \mono{Int#} (unboxed integer)
in which it is defined: \hs{data Int = I# Int#}. Computations on \mono{Int#}s can be evaluated much more cheaply since such values can be
stored on the stack.

Knowing this, we can opt to create a worker that does add state to the computation, but changes the types to their unboxed counterparts.
Naturally the wrapper operation is then simply the constructor of the lazy type \hs{I#}. Let us consider the example of the recursive \mono{triangular} function which
given a number \mono{n} returns the sum of all numbers from 1 to \mono{n}:

\begin{listing}[H]
\begin{minted}[linenos]{haskell}
triangular :: Int -> Int
triangular 0 = 0
triangular n = n + triangular (n-1)
\end{minted}
\end{listing}

Although an all-knowing compiler could have determined that the result of \mono{triangular} can simply be calculated numerically in constant time as $\frac{n^2 + n}{2}$,
we can still trust GHC to infer an important property about the function using strictness analysis. Namely: \mono{triangular} does not produce any intermediate results that can
used without evaluating the whole of \mono{triangular}. That is: if you are willing to spent any amount of time on \mono{triangular}, you might as well calculate the whole thing.
Thus GHC decides to rewrite the function using a strict worker, removing the need for many dubious allocations:

\begin{listing}[H]
\begin{minted}[linenos]{haskell}
wtriangular :: Int# -> Int#
wtriangular 0 = 0
wtriangular n = GHC.Prim.+# n (wtriangular (GHC.Prim.-# n 1))

triangular :: Int -> Int
triangular (I# w) = I# (wtriangular w)
\end{minted}
\end{listing}

To get a feeling for the difference in performance, we can compare the runtime of the original function with the worker/wrapper version:

\begin{table}[H]
\begin{tabular}{|l|l|}
\textbf{Snippet} (compiled with -O0) & \textbf{Result of \mono{time} with input \mono{10000000}}\\
\hline \\
\mono{original} & 0,10s user 0,03s system 99\% cpu 0,132 total \\
\mono{transformed} & 0,01s user 0,01s system 98\% cpu 0,023 total \\
\end{tabular}
\caption{The runtime of both version of \mono{triangular}. We can see that the worker/wrapper tranformation has increased runtime performance by a significant factor.}
\end{table}

It might not be immediately intuitive why the performance differs so drastically and where exactly these seemingly evil allocation occur.
The ridiculousness of the original function becomes apparent when we consider a C implementation with the same allocation behavior. 
Although lacking in laziness, we can consider an \mono{Int} to map to a \mono{long*} (pointing to heap allocated memory) 
and an \mono{Int#} to map to a plain \mono{long}.


\begin{minted}[linenos]{c}
#include <stdio.h>
#include <stdlib.h>

// utility function for heap allocating a 64 bit integer
long* alloc_long() { return (long*)malloc(sizeof(long)); }

long* triangular(const long* n_ptr) {
  // allocate the return value
  long* ret = alloc_long();
  // derefence the pointer into a value
  long n = *n_ptr;
  if (n==0) { *ret = 0; } else {
    // allocate a new pointer with which to call the function recursively
    long* inp_ptr = alloc_long();
    *inp_ptr = n-1 ;
    *ret = n + *triangular(inp_ptr);
  }
  return ret;
}
\end{minted}

Rest assured that any C programmer suggesting this implementation would get some very weird looks.

\subsection{Analysis transformations}

Consider again the Core ADT given in \cref{section:background:core_lang}, it was mentioned that the \mono{Var}
type is used to represent variables and their various properties. We look into the one field of \mono{Var} that
is dynamic during the core2core pipeline: \mono{IdInfo}.

This data type follows the concept of weakening of information, i.e. the information is never incorrect but
may be less precise or even missing. Furthermore, the \mono{IdInfo} may differ for different \mono{Var} instances
that refer to the same variable.

It is as of \mono{ghc-9.4.2} defined as:

\begin{listing}[H]
\begin{minted}[linenos,fontsize=\footnotesize]{haskell}
data IdInfo
  = IdInfo {
        ruleInfo        :: RuleInfo,
        -- ^ Specialisations of the 'Id's function which exist.
        -- See Note [Specialisations and RULES in IdInfo]
        realUnfoldingInfo   :: Unfolding,
        -- ^ The 'Id's unfolding
        inlinePragInfo  :: InlinePragma,
        -- ^ Any inline pragma attached to the 'Id'
        occInfo         :: OccInfo,
        -- ^ How the 'Id' occurs in the program
        dmdSigInfo      :: DmdSig,
        -- ^ A strictness signature. Digests how a function uses its arguments
        -- if applied to at least 'arityInfo' arguments.
        cprSigInfo      :: CprSig,
        -- ^ Information on whether the function will ultimately return a
        -- freshly allocated constructor.
        demandInfo      :: Demand,
        -- ^ ID demand information
        bitfield        :: {-# UNPACK #-} !BitField,
        -- ^ Bitfield packs CafInfo, OneShotInfo, arity info, LevityInfo, and
        -- call arity info in one 64-bit word. Packing these fields reduces size
        -- of `IdInfo` from 12 words to 7 words and reduces residency by almost
        -- 4% in some programs. See #17497 and associated MR.
        --
        -- See documentation of the getters for what these packed fields mean.
        lfInfo          :: !(Maybe LambdaFormInfo),

        -- See documentation of the getters for what these packed fields mean.
        tagSig          :: !(Maybe TagSig)
    }
\end{minted}
\end{listing}

Through non-structure-altering transformations like \textit{Demamd Analysis} and \textit{Strictness Analysis} the \mono{IdInfo}
record is updated accordingly. This information may then be used by future transformations that can only optimise safely or
productively under certain circumstances (remember the disastrous work duplication in \cref{section:background:inlining}?).

The information contained in \mono{IdInfo} is a major source of complexity when it comes to comprehending Core. Consider
for example this `pretty'-printed \mono{IdInfo} instance:

\begin{listing}[H]
\begin{minted}[linenos,fontsize=\footnotesize]{haskell}
[GblId,
 Arity=4,
 Str=<L,U(U(U(U(C(C1(U)),A,C(C1(U)),C(U),A,A,C(U)),A,A),1*U(A,C(C1(U)),A,A),A,A,A,A,A),U(A,A,C(U),...etc.
 Unf=Unf{Src=<vanilla>, TopLvl=True, Value=True, ConLike=True,
         WorkFree=True, Expandable=True, Guidance=IF_ARGS [50 0 0 0] 632 0}]
\end{minted}
\end{listing}

We can quickly learn of few things about the variable it describes. First,  it is apparently a function that has
an arity of 4 (i.e. it takes 4 arguments). Secondly, we obtain some of the magic heuristic values involved with inlining
(also known \textit{unfolding} hence \mono{Unf}). However, if we want to diagnose why for example this variable was evaluated
lazily even though it is always used exactly once, we would have to decode the strictness signature under \mono{Str}.
Currently, the best resource for decoding this would be to read the comments in the GHC source code itself.

\section{The fusion system}
\label{section:background:fusion}

Fusion --- the process of combining multiple operations over a structure into a single operation --- is in GHC
implemented using the \textit{build/foldr} idiom. This system was developed as an improvement on deforestation 
(Waddler \cite{WADLER1990231}) which has shortcomings when it comes to recursive functions \cite{shortcut_fusion}.

We borrow an example from Seo \cite{shortcut_fusion_blog} to illustrate how the build/foldr system works. Consider
the example --- coincidentally very similar to \mono{triangular} (\cref{section:background:worker_wrapper}) --- of
summing a list of numbers:

\begin{listing}[H]
\begin{minted}[linenos]{haskell}
foldr (+1) 0 [1..10]
\end{minted}
\end{listing}

Very concise indeed but its execution would be much slower than say, an imperative for loop, because we first create
a list and then consume it right away. If we look inside the definition of \mono{[1..10]} we find the function \mono{from}:

\begin{listing}[H]
\begin{minted}[linenos]{haskell}
from :: (Ord a, Num a) => a -> a -> [a]
from a b = if a > b
           then []
           else a : from (a + 1) b
\end{minted}
\end{listing}

Which is itself a specialisation for lists. We can write a more generic catamorphism that takes any duo of constructors
of type \mono{a -> b -> b} and \mono{b} respectively (previously \mono{:} and \mono{[]}):

\begin{listing}[H]
\begin{minted}[linenos]{haskell}
from' :: (Ord a, Num a) => a -> a -> (a -> b -> b) -> b -> b
from' a b = \c n -> if a > b
                    then n
                    else c a (from' (a + 1) b c n)
\end{minted}
\end{listing}

The glue between \mono{from} and \mono{from'} is the \textit{build} function, which requires the \mono{Rank2Types} language extension
and re-specialises these generic functions back to lists.

\begin{listing}[H]
\begin{minted}[linenos]{haskell}
build :: forall a. (forall b. (a -> b -> b) -> b -> b) -> [a]
build g = g (:) []

from a b = build (from' a b )
\end{minted}
\end{listing}

Thus far it seems we have only introduced more complexity without any apparent benefit. 
However, this is the point that we run into the key idea: \mono{build} is the antithesis of \mono{foldr} such that the following holds:

\begin{listing}[H]
\begin{minted}[linenos]{haskell}
foldr k z (build g) = g k z
\end{minted}
\end{listing}

And this also gives us the main reductive rewrite rule to make this system work.
So to summarize, we have shown that generalizing lists to list producing functions to catamorphisms
over list like arguments, produces a common interface exposing the elimination of code. More concretely,
because we keep \mono{g} as a generic function, we can choose to give it the arguments of the consequent
\mono{foldr} directly and thus eliminate the intermediate list. This is again a situation like the assembly line
workers analogy from \cref{section:introduction:inspection_testing}; It is very productive to communicate about
functions taking and producing lists, but actually doing so in a chained context is like wrapping and unwrapping
boxes at every step.

Analogies aside, observe how our original expression can now completely be reduced to a single integer value
by rewriting \mono{from} with this common interface.

\begin{listing}[H]
\begin{minted}[linenos]{haskell}
foldr (+) 0 (from 1 10)
-- { inline from }
foldr (+) 0 (build (from' 1 10))
-- { apply rewrite rule }
from' 1 10 (+) 0
-- { inline from' }
\c n -> (if 1 > 10
            then n
            else c 1 (from' 2 10 c n)) (+) 0
-- { beta reduce }
if 1 > 10
   then 0
   else 1 + (from' 2 10 (+) 0)
-- { repeat until base case }
1 + 2 + ... + 9 + 10 + 0
-- { constant fold }
55
\end{minted}
\end{listing}

But what if there is no \mono{foldr} in the expression? What if we use simpler functions like \mono{map} or \mono{filter}?
Well, as most introductory Haskell courses are likely to tell you, most list functions can be defined in terms of \mono{foldr}.
Thus, we can similarly use rewrite rules to map all these common list functions to a combination \mono{build} and \mono{foldr}.
This idea also relieves us from the burden of having to define rewrite rules for all the million possible 
combination of lists operations:

\begin{listing}[H]
\begin{minted}[linenos]{haskell}
map f xs    = build (\ c n -> foldr (\ a b -> c (f a) b) n xs)
filter f xs = build (\ c n -> foldr (\ a b -> if f a then c a b else b) n xs)
xs ++ ys    = build (\ c n -> foldr c (foldr c n ys) xs)
concat xs   = build (\ c n -> foldr (\ x y -> foldr c y x) n xs)
repeat x    = build (\ c n -> let r = c x r in r)
zip xs ys   = build (\ c n -> 
    let zip' (x:xs) (y:ys) = c (x,y) (zip' xs ys)
        zip' _ _ = n
    in  zip' xs ys)
[]          = build (\ c n -> n)
x:xs        = build (\ c n -> c x (foldr c n xs))
\end{minted}
\end{listing}

We can see some these functions at work when looking at the definition of \mono{unlines} and its subsequent
reductions through the build/foldr system:

\begin{listing}[H]
\begin{minted}[linenos]{haskell}
unlines :: [String] -> String
unlines ls = concat (map (\l -> l ++ ['\n']) ls)
-- { rewrite concat and map }
unlines ls = build
  (\c0 n0 -> foldr (\xs b -> foldr c0 b xs) n0 ( build
    (\c1 n1 -> foldr (\l t -> c1 (build
      (\c2 n2 -> foldr c2 ( foldr c2 n2 ( build
        (\c3 n3 -> c3 '\n' n3))) l )) t) n1 ls )))
-- { apply rewrite rule foldr/build }
unlines ls = build
  (\c0 n0 ->
    (\c1 n1 -> foldr (\l t -> c1 (build
      (\c2 n2 -> foldr c2 (
        (\c3 n3 -> c3 '\n' n3) c2 n2 ) l)) t) n1 ls)
          (\xs b -> foldr c0 b xs) n0)
-- { beta reduce }
unlines ls = build
  (\ c0 n0 -> foldr (\l t -> foldr c0 t( build
    (\c2 n2 -> foldr c2 (c2 '\n' n2) l))) n0 ls)
-- { apply rewrite rule foldr/build }
unlines ls = build
  (\c0 n0 -> foldr (\l b -> foldr c0 (c0 '\n' b) l) n ls)
-- { inline build }
unlines ls = foldr (\l b -> foldr (:) ('\n' : b) l) [ ] ls
-- { inline foldr }
unlines ls = h ls
    where h [] = []
          h (l:ls) = g l
              where g [] = '\n' : h ls
                    g (x:xs) = x : g xs
\end{minted}
\end{listing}

What we end up with is the most efficient implementation of \mono{unlines} that we could possibly write by hand \cite{shortcut_fusion}.
Because the list \text{ls} is the input of the function we cannot remove it all together, but keep in mind that when inlining call to this function
at the use site, foldr/build fusion may apply again. 

\section{Contemporary Haskell comprehension}

\subsection{Communicating in Core}
\label{section:communicating_core}

We are not pioneers discovering the land of Core inspection. Since its inception, Core has
been used to communicate about programs and compiler interactions. Sifting through open issues on the
GHC compiler itself, one quickly comes across discussions elaborated by Core snippets. Consider issue
\href{https://gitlab.haskell.org/ghc/ghc/-/issues/22207}{\#22207} titled `\textit{bytestring Builder performance regressions after 9.2.3}' for example.
\hfill \break

\textit{While testing the performance characteristics of a bytestring patch meant to mitigate withForeignPtr-related performance regressions,
it was noticed that several of our Builder-related benchmarks have also regressed seriously for unrelated reasons.
The worst offender is byteStringHex, which on my machine runs about 10 times slower and allocates 21 times as much when using ghc-9.2.4 or ghc-9.4.2
as it did when using ghc-9.2.3. Here's a small program that can demonstrate this slowdown:}
\hfill \break

The author then provides two snippets containing the final Core representation of \mono{byteStringHex} as produced by the two different GHC version.
Each of these documents contain around 400 lines of unhighlighted code annotated with all available information. And while having all available information
sounds like a good thing (and it is in a way) it poses a serious practicality issue.
Namely: it is exceedingly difficult for a human to read and comprehend a certain aspect of the AST while having to filter out another.
Not to mention, it solidifies reading Core as an activity reserved for only the most expert Haskell developers by scaring others away with
a steep barrier to entry.

\subsection{Current tooling}

\subsubsection{GHC itself}
\label{section:background:ghc}


Core snippets of your program can easily be coerced out of GHC. The most information you can get
is the Core AST at each pass of the optimisation pipeline by using \mono{-ddump-core2core}.
To reduce the signal-to-noise ratio of Core snippets, one can use any number of suppression options.
It is common to suppress type arguments and type applications for example. These are largely
uninteresting because Core is only typed to facilitate the correctness preserving checks that \mono{core-lint} performs
and do not affect the compilation in any way.

As can be read in the GHC documentation, the following suppression flags are available to help to tame the beast.

\begin{table}[H]
  \begin{tabular}{p{0.35\linewidth}|p{0.65\linewidth}}
  \textbf{GHC flag} & \textbf{Effect on Core printing} \\
  \hline
           & \\
  \mono{-dsuppress-all} & In dumps, suppress everything (except for uniques) that is suppressible.  \\
  \mono{-dsuppress-coercions} & Suppress the printing of coercions in Core dumps to make them shorter  \\
  \mono{-dsuppress-core-sizes} & Suppress the printing of core size stats per binding (since 9.4)  \\
  \mono{-dsuppress-idinfo} & Suppress extended information about identifiers where they are bound  \\
  \mono{-dsuppress-module-prefixes} & Suppress the printing of module qualification prefixes  \\
  \mono{-dsuppress-ticks} & Suppress "ticks" in the pretty-printer output.  \\
  \mono{-dsuppress-timestamps} & Suppress timestamps in dumps  \\
  \mono{-dsuppress-type-applications} & Suppress type applications  \\
  \mono{-dsuppress-type-signatures} & Suppress type signatures  \\
  \mono{-dsuppress-unfoldings} & Suppress the printing of the stable unfolding of a variable at its binding site  \\
  \mono{-dsuppress-uniques} & Suppress the printing of uniques in debug output (easier to use diff)  \\
  \mono{-dsuppress-var-kinds} & Suppress the printing of variable kinds\\
\end{tabular}
\end{table}

We can show how these suppression options greatly improve the readability of Core snippets by comparing the
desugared (unoptimized) Core of \mono{quicksort} with and without the \mono{-dsuppress-all} flags.

First the source:

\begin{listing}[H]
\begin{minted}[linenos]{haskell}
quicksort :: Ord a => [a] -> [a]
quicksort [] = []
quicksort (x:xs) = quicksort (filter (< x) xs) ++ [x] ++ quicksort (filter (>= x) xs)
\end{minted}
\label{code:quicksort_src}
\end{listing}

The desugared Core without suppression:

\begin{listing}[H]
\begin{minted}[linenos,fontsize=\footnotesize]{text}
-- RHS size: {terms: 55, types: 47, coercions: 0, joins: 0/10}
quicksort :: forall a. Ord a => [a] -> [a]
[LclIdX]
quicksort
  = \ (@ a_a1pH) ($dOrd_a1pJ :: Ord a_a1pH) ->
      let {
        $dOrd_a1w8 :: Ord a_a1pH
        [LclId]
        $dOrd_a1w8 = $dOrd_a1pJ } in
      let {
        $dOrd_a1w5 :: Ord a_a1pH
        [LclId]
        $dOrd_a1w5 = $dOrd_a1pJ } in
      \ (ds_d1wq :: [a_a1pH]) ->
        case ds_d1wq of wild_00 {
          [] -> ghc-prim-0.6.1:GHC.Types.[] @ a_a1pH;
          : x_a1hP xs_a1hQ ->
            letrec {
              greater_a1hS :: [a_a1pH]
              [LclId]
              greater_a1hS
                = let {
                    $dOrd_a1pQ :: Ord a_a1pH
                    [LclId]
                    $dOrd_a1pQ = $dOrd_a1pJ } in
                  letrec {
                    greater_a1pT :: [a_a1pH]
                    [LclId]
                    greater_a1pT
                      = filter
                          @ a_a1pH
                          (let {
                             ds_d1wF :: a_a1pH
                             [LclId]
                             ds_d1wF = x_a1hP } in
                           \ (ds_d1wE :: a_a1pH) -> > @ a_a1pH $dOrd_a1pQ ds_d1wE ds_d1wF)
                          xs_a1hQ; } in
                  greater_a1pT; } in
            letrec {
              lesser_a1hR :: [a_a1pH]
              [LclId]
              lesser_a1hR
                = let {
                    $dOrd_a1vV :: Ord a_a1pH
                    [LclId]
                    $dOrd_a1vV = $dOrd_a1pJ } in
                  letrec {
                    lesser_a1vY :: [a_a1pH]
                    [LclId]
                    lesser_a1vY
                      = filter
                          @ a_a1pH
                          (let {
                             ds_d1wD :: a_a1pH
                             [LclId]
                             ds_d1wD = x_a1hP } in
                           \ (ds_d1wC :: a_a1pH) -> < @ a_a1pH $dOrd_a1vV ds_d1wC ds_d1wD)
                          xs_a1hQ; } in
                  lesser_a1vY; } in
            ++
              @ a_a1pH
              (quicksort @ a_a1pH $dOrd_a1w5 lesser_a1hR)
              (++
                 @ a_a1pH
                 (GHC.Base.build
                    @ a_a1pH
                    (\ (@ a_d1wx)
                       (c_d1wy :: a_a1pH -> a_d1wx -> a_d1wx)
                       (n_d1wz :: a_d1wx) ->
                       c_d1wy x_a1hP n_d1wz))
                 (quicksort @ a_a1pH $dOrd_a1w8 greater_a1hS))
        }
\end{minted}
\label{code:quicksort_core_raw}
\end{listing}

The same desugared Core representation with the \mono{-dsuppress-all} flag:

\begin{listing}[H]
\begin{minted}[linenos,fontsize=\footnotesize]{text}
-- RHS size: {terms: 55, types: 47, coercions: 0, joins: 0/10}
quicksort
  = \ @ a_a1pH $dOrd_a1pJ ->
      let { $dOrd_a1w8 = $dOrd_a1pJ } in
      let { $dOrd_a1w5 = $dOrd_a1pJ } in
      \ ds_d1wq ->
        case ds_d1wq of wild_00 {
          [] -> [];
          : x_a1hP xs_a1hQ ->
            letrec {
              greater_a1hS
                = let { $dOrd_a1pQ = $dOrd_a1pJ } in
                  letrec {
                    greater_a1pT
                      = filter
                          (let { ds_d1wF = x_a1hP } in
                           \ ds_d1wE -> > $dOrd_a1pQ ds_d1wE ds_d1wF)
                          xs_a1hQ; } in
                  greater_a1pT; } in
            letrec {
              lesser_a1hR
                = let { $dOrd_a1vV = $dOrd_a1pJ } in
                  letrec {
                    lesser_a1vY
                      = filter
                          (let { ds_d1wD = x_a1hP } in
                           \ ds_d1wC -> < $dOrd_a1vV ds_d1wC ds_d1wD)
                          xs_a1hQ; } in
                  lesser_a1vY; } in
            ++
              (quicksort $dOrd_a1w5 lesser_a1hR)
              (++
                 (build (\ @ a_d1wx c_d1wy n_d1wz -> c_d1wy x_a1hP n_d1wz))
                 (quicksort $dOrd_a1w8 greater_a1hS))
        }

\end{minted}
\label{code:quicksort_core_dsuppress}
\end{listing}

A drastic improvement for sure, but there are still some things to be left desired.
A simpler language like Core generally needs more code to express the same thing, we can thus expect
to generate more data than the original Haskell code. Moreover, should you be interested the state off
the program at each of the intermediate steps, you can expect to see about 20x more data still.
Unless you know exactly what to search for, this begs for a more ergonomic, filtered view of the data.

\subsubsection{GHC Plugins}

GHC --- being also a playground for academics and enthusiasts alike --- is extremely flexible when it comes to
altering its functionality. Using the now well established plugin interface, one is able to hook into almost
any operation of the front- and midsection of the compiler. One such place is managing the core2core passes that
will be run on the current module. This point of customization can be used to intersperse each core2core pass with an
identity transformation that smuggles away a copy of the AST in its full form.

One such existing plugin is \mono{ghc-dump} \cite{ghc_dump}. Besides extracting intermediate ASTs, it defines an
auxiliary Core definition to which it provides a GHC version agnostic mapping. This has the increased benefit of 
being able to directly compare snapshots from different GHC versions; A not uncommon task as discussed in \cref{section:communicating_core}.
And while certainly being an improvement over plain text representation, we believe exploring and comparing such
dumps requires a more rich interface.
