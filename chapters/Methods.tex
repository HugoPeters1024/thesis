\chapter{Methods}

\section{Requirements and architecture}

\begin{itemize}
  \item GHC version agnostic (to where reasonable i.e. 8.4 onwards)
  \item Simple and non invase steps to create dumps
  \item Cross-platform ability to explore dumps
  \item Not everything needs to be supported (think unfoldings) but should be extendible in the future
\end{itemize}

We envisioned a high degree of interactability with the snapshots of the intermediate ASTs. To realise this
in a cross-platform fashion, we decided to use a browser based frontend application. Because the concept of
mutually recursive algebraic datatypes are very pervasive in the Core AST, we felt it would be helpful
if the frontend language had first class support for this. This naturally led to us to Elm, a functional language
that compiles to Javscript \cite{elm_lang}.

\section{Creating the GHC plugin}

By far the most logical and convenient method for obtaining all the available information in the AST it to capture
it directly using a GHC core2core plugin that intersperses each transformation with an identity transformation that
saves a JSON representation of the AST to a file as a side effect. 

\subsection{Capturing the information}

If we wish to support multiple recent versions of GHC we need to deal with the fact that the Core ADT has undergone
a few minor changes and additions. We believe that the solution is to create some auxiliary definition to which we can
map various versions of the Core ADT. This was done very efficiently by building upon the existing \mono{ghc-dump} package,
which already implemented such an ADT as well as a version agnostic conversion module with the help of \mono{min_version_GHC}
macro statements \cite{ghc_dump}. 

What \mono{ghc-dump} also intelligently addresses, is the issue of possible infinite recursion
in the Core AST through the \mono{CoreExpr} inside the unfolding of a \mono{CoreBndr}. If we demote each
call site to a simple identifier, we obtain a finite representation that we can later reconstruct by traversing
the AST with an environment.

\subsection{Globally unique variables}

It is not strictly necessary for variable names in a program to be unique. A variable name always
references the nearest binding site. This holds true for Core as well, but is not very convenient
when we want to analyze a certain variable in isolation. After all, naked variables are ambiguous without a given
environment. Consider the definition of tail:

\begin{listing}[H]
\begin{minted}{haskell}
tail xs = case xs of
  x:xs -> xs
  _    -> error "tail of empty list"
\end{minted}
\end{listing}

We cannot simply refer to the variable \mono{xs} as that name has two different binding sites.
We solve this by running a \textit{uniqify} pass before each snapshot that freshens all duplicate uniques in the entire
module after each core2core transformation. This gives us the ability to refer to a binding site and its usages unambiguously using simply an integer.
The big idea here is that any viewing logic is completely decoupled from binding semantics:

\begin{listing}[H]
\begin{minted}{haskell}
tail xs_0 = case xs_0 of
  x_1:xs_2 -> xs_2
  _        -> error "tail of empty list"
\end{minted}
\end{listing}

It is possible to omit the numbered suffixes when displaying the AST, but internally it is very useful to be able to make
this distinction without any further effort.

\subsection{Detecting changes}

If a module is of a slightly larger size, it becomes difficult to spot the changes made by a certain
transformation, if there even are any. To address this, we decided to develop a feature that allows for
the filtering of code that remains unchanged. Let us define what unchanged means in this context. It is
important to make the subtle distinction between syntactic equivalence and $\alpha$-equivalence. The difference
is that the latter is agnostic to the names of variables, as long as they refer to the same binding site.

We can quickly solve the decision problem of syntactic equality by calculating a hash of an expression beforehand
and simply checking for equality of this hash value. We considered using recent improvements of full sub expression
matching \cite{hashing_mod_alpha}, but decided against it as it was not clear how to effectively present the results
nor did it rarely prove useful to isolate changes in the AST as they were rarely local to begin with.
Instead, we opted for a far simpler approach where we only hash the toplevel functions, and provide a more crude
option to hide any toplevel definitions that have not changed at all.

All in all, we still recommend that issues are attempted to be reproduced in small modules as the amount of
noise can quickly become overwhelming despite change detection.

\section{Creating the frontend application}

\subsection{Reproducing the AST}
It would have been extremely tedious to have to constantly maintain a Core ADT in Elm along with a JSON
parser that is compatible with the JSON output of the Haskell plugin. Luckily, we were able to use the
\mono{haskell-to-elm} \cite{haskell_to_elm} package to automatically derive all the needed boilerplate code.

For example the \mono{Alt} datatype (i.e. an arm of a case expression) is defined as follows in Haskell:

\begin{listing}[H]
\begin{minted}{haskell}
data Alt = Alt
    { altCon :: AltCon
    , altBinders :: [Binder]
    , altRHS :: Expr
    }
    deriving (Generic)
\end{minted}
\end{listing}

The corresponding, autogenerated, Elm definition becomes:

\begin{listing}[H]
\begin{minted}{elm}
type alias Alt =
    { altCon : AltCon
    , altBinders : List Binder
    , altRHS : Expr
    }

altDecoder : Json.Decode.Decoder Alt
altDecoder =
    Json.Decode.succeed Alt |>
    Json.Decode.Pipeline.required "altCon" altConDecoder |>
    Json.Decode.Pipeline.required "altBinders" (Json.Decode.list binderDecoder) |>
    Json.Decode.Pipeline.required "altRHS" exprDecoder
\end{minted}
\end{listing}

Conviently, it would be very easy in the future to extend the Haskell ADT with additional information
because it is the single source of truth; The Elm ADT and JSON machinery can then be regenerated with a single command.

\subsection{Pretty printing}

It certainly should be considered useful to display the Core in exactly the same way that
GHC does. After all, this is what programmers are currently already used to and its design
has been given a lot of thought. However, we felt a need to first create a separate representation
that is tailored to those who have not seen Core before.

Haskell programmers that care enough to look at the interaction with the compiler are likely to be
avid readers of at least basic Haskell syntax. Therefore we decided to create a pretty printer that attempts
to be as similar to Haskell source as possible. Just like GHC, we used pretty printing the method developed 
by Waddler \cite{prettier_printer}, implemented in Elm using \mono{elm-pretty-printer} \cite{prettier_printer_elm}.



\subsection{Including the source}
Using python3-pygments to generate highlighted html that can be hacked into the DOM directly

\subsection{Performance considerations}
Using the lazy html utility for the code pretty printer to ensure interactive
update times for other DOM changes.

