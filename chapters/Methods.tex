\chapter{Methods}

\section{Defining the requirements}

\begin{itemize}
  \item GHC version agnostic (to where reasonable i.e. 8.4 onwards)
  \item Simple and non invase steps to create dumps
  \item Cross platform ability to explore dumps
  \item Not everything needs to be supported (think unfoldings) but should be extendible in the future
\end{itemize}

\section{Creating the GHC plugin}

\subsection{Capturing the information}
All about what GHC encodes in core (IdInfo etc.)

\subsection{Globally unique variables}

It is not strictly necessary for variable names in a program to be unique. A variable name always
references the nearest binding site. This holds true for Core as well, but is not very convenient
when we want to analyze a certain variable. After all, naked variables are ambiguous without a given
environment. We solve this by freshening all duplicate uniques in the entire module after each core2core
transformation. This gives us the ability to refer to a binding site and its usages unambiguously using simply an integer.
The big idea here is that any viewing logic no longer needs to deal with binding semantics in any way.

\subsection{Hashing toplevel definitions}
Full fletched expression comparison is hard, but comparing only toplevel definitions by their hash
is a manageable compromise that still allows eliminating large chunks of noise.

\section{Creating the frontend application}
All about ELM, JSON, and why it may be suboptimal

\subsection{Communication}
Deriving ToJson and Elm code from Haskell types

\subsection{Pretty printing}
Creating a beginner-friendly core representation

\subsection{Including the source}
Using python3-pygments to generate highlighted html that can be hacked into the DOM directly

\subsection{Performance considerations}
Using the lazy html utility for the code pretty printer to ensure interactive
update times for other DOM changes.
