\chapter{Introduction}

\section{GHC, an optimising compiler}

Haskell is a high level language designed to write highly composable functions.
This naturally encourages programmers to write code that
is not particularly fast to evaluate. An extremely common example is the composition
of list operations. Consider the function \mono{halves} which divides each element
in a list of \mono{Int}s in halve, discarding those that are not a multiple of 2.

\begin{listing}[H]
\begin{minted}[linenos]{haskell}
halves :: [Int] -> [Int]
halves = map (`div` 2) . filter even
\end{minted}
\end{listing}

Both \mono{map} and \mono{filter} are defined as a loop over an input list, producing a new list as output. 
If code was generated for \mono{halves} in its current form, it would produce two loops as well as allocate
an intermediate list. This is unnecessary extra work and incurs more allocation costs because
it would be possible to rewrite the function to circumvent this issue:


\begin{listing}[H]
\begin{minted}[linenos]{haskell}
halves_fast :: [Int] -> [Int]
halves_fast [] = []
halves_fast (x:xs) = 
  let 
    tl = halves_fast xs 
  in if even x 
     then (x `div` 2):tl
     else tl
\end{minted}
\end{listing}

However, requiring the programmer to do such rewrites manually tragically undermines
the benefits of this compositional style of programming. Code simply would be harder to read, write and
maintain and likely to be less correct as a result. 

Luckily, GHC does address this issue - and many others - with an extensive set of optimisation transformations.
This particular program will benefit greatly from the fusion system which specifically deals with removing
the intermediate lists. This is a well-established optimisation that is also referred to as deforestation \cite{WADLER1990231}.
As a result, compiling with these optimisations enabled will result in a \textbf{syntactically} equivalent
definition for \mono{halves} and \mono{halves_fast}! 

But this poses a question of trust; No compiler is every perfect, so how can we be sure that our code is
correctly optimised and will continue to be in the future? Furthermore, if optimisations are missed, how
can we diagnose the root of the problem and explain what went wrong?

\section{The cascade effect}

Optimisation transformations are applied in a certain order, giving rise to the \textit{cascade effect} \cite{haskell_optimisations_1997}.
This effect refers to the dramatic consequences that the order of transformations can have. We will study an example that
showcases a problematic tug-of-war between two optimisations: (1) inlining and (2) rewrite rules.

\subsection{The inlining transformation}

Inlining is arguably the most important optimisation for any functional language. Coincidentally,
it is also one of the simplest transformations to implement and comprehend. To reveal
why it is such a staple, we must consider it in conjunction with \textbeta-reduction. 
These transformations are fairly simply defined as:

\begin{listing}[H]
\begin{minted}[linenos]{haskell}
-- inlining, knowing that x = x' from some binding site
e                 -inline->      e[x := x']

-- beta reduction:
(\x -> e) a       -ß->           e[x := a]
\end{minted}
\end{listing}

We can apply this in a Haskell context to get a more familiar perspective. 
Consider the boolean negation function \mono{not} and a basic usage thereof:

\begin{listing}[H]
\begin{minted}[linenos]{haskell}
not :: Bool -> Bool
not = \x -> if x then False else True

top = True
bottom = not top
\end{minted}
\end{listing}

In the case of \hs{bottom} function we can elect to inline both \mono{not} and \mono{top}
in its body. This gives us 

\begin{listing}[H]
\begin{minted}[]{haskell}
bottom = (\x -> if x then False else True) True
\end{minted}
\end{listing}

We can now perform \textbeta-reduction by taking the body of the lambda function
and substituting \mono{x} by \mono{True}:

\begin{listing}[H]
\begin{minted}[]{haskell}
bottom = if True then False else True
\end{minted}
\end{listing}

Finally, we eliminate the \mono{if} statement by evaluation. Knowing that it always takes the first branch, we
get the final result of:

\begin{listing}[H]
\begin{minted}[]{haskell}
bottom = True
\end{minted}
\end{listing}

Thanks to this transformation duo, we have reduced an expression to a mere literal, 
eliminating any runtime cost. Because the inlining transformation so commonly goes hand
in hand with \textbeta-reduction we will -- for brevity's sake -- from now on presume that \textbeta-reduction,
wherever relevant, is also applied when we say that an inlining transformation has taken place.

In a sense we were lucky here that the body of the \mono{not} function reduced so completely. But in reality
it can be the case that function definitions are quite large. Inlining will then still eliminate the need for a function call,
but it does come at the cost of larger code sizes through code duplication.
GHC uses quite a few crude heuristics to help with weighing this trade-off \cite{haskell_optimisations_1997}.
Interestingly, these heuristics have not undergone much reconsideration until quite recently \cite{inling_magic_numbers}. 
Suffice to say that it is not trivial to predict whether a function will be inlined and the performance implications thereof.

\subsubsection{Laziness and thunks}
Being a lazy language, Haskell can profit from inlining in a secondary way as well. A well-established
language feature is the let expression, a syntactically nice way to bind extra definitions. Consider a function
that decrypts the password of a user but only if it is the admin.

\begin{listing}[H]
\begin{minted}[linenos]{haskell}
getAdminPassword :: User -> Maybe String
getAdminPassword user = 
  let decrypted :: String
      decrypted = decryptSHA1 (password user)

  in case (username user) of
    "admin" -> Just decrypted
    _       -> Nothing
\end{minted}
\end{listing}

Obviously the \mono{decryptSHA1} is going to be an extremely costly function that under no
circumstances we wish to be evaluated unless absolutely necessary. Luckily, because Haskell is a
lazy language the let assignment does not actually evaluate anything; it only allocates a so-called
\textit{thunk}. Such a thunk represents an expression that is yet to be evaluated. Evaluation is
in fact deferred to the point where the thunk is actually needed (which is never if the username is
not \hs{"admin"}). So, because of this property \mono{getAdminPassword} is actually quite effective 
at avoiding unnecessary work. 

Yet it can still be improved upon. While the let expression might not do any extra work it does require
this thunk allocation, which is not free. But because our let bound variable \mono{decrypted} 
is used exactly once, we can inline it at the only call-site without running the risk of duplicating 
the expensive operation. This yields the ever so slightly more performant definition:

\begin{listing}[H]
\begin{minted}[linenos]{haskell}
getAdminPassword :: User -> Maybe String
getAdminPassword user = case (username user) of
    "admin" -> Just (decryptSHA1 (password user))
    _       -> Nothing
\end{minted}
\end{listing}

\subsection{Rewrite rules}
\label{section:introduction:rewrite_rules}

We've seen how generic program transformation may improve performance. However, GHC can only use relatively
shallow reasoning as to not jeopardize the correctness of transformations or explode compile times. 
The programmer on the other hand, may have much more in-depth knowledge about the domain of the program 
and its intended behavior. \cite{playing_by_the_rules}.
Programmers can leverage this knowledge by defining so-called \textit{rewrite rules}. They inform the compiler of
substitutions that are not obvious or strictly speaking even correct.  

Consider the binary tree datatype \mono{Tree} with along with a the higher order \mono{mapTree} functions that
facilities transforming the values contained in the \mono{Leaf}s. We then use \mono{mapTree} to compose two
traversals with an addition in the function \mono{addFive}. Obviously that function is rather contrived as
an example of something that someone would write, but the pattern may very well show up during the transformation pipeline.


\begin{listing}[H]
\begin{minted}[linenos]{haskell}
data Tree a = Leaf a | Node (Tree a) (Tree a) deriving Show

mapTree :: (a -> b) -> Tree a -> Tree b
mapTree f (Leaf x) = Leaf (f x)
mapTree f (Node lhs rhs) = Node (mapTree f lhs) (mapTree f rhs)

addFive :: Tree Int -> Tree Int
addFive = mapTree (+1) . mapTree (+4)
\end{minted}
\end{listing}

By now it should be clear why \mono{addFive} is non-optimal; it has a superfluous traversal and allocates an
intermediate structure. \hs{mapTree (+5)} is the far superior equivalent. However, for GHC to infer this fact
it has to perform a too complicated and generic analysis. But we can add the rewrite rule \textit{mapTree/mapTree}
to convince GHC that consecutive applications of \mono{mapTree} are allowed to fuse:

\begin{minted}[linenos]{haskell}
{-# Rules
   "mapTree/mapTree"   forall f g.   mapTree f . mapTree g    = mapTree (f. g)   ;
#-}
\end{minted}

But as it turns out, \textit{mapTree/mapTree} never fired for a reason that is not immediately obvious.

\subsubsection{A note on common abstraction}

A clever reader might have noticed that \mono{mapTree} is a perfect candidate for an implementation of \mono{fmap}
as part of the \mono{Functor} typeclass: 

\begin{minted}[linenos]{haskell}
instance Functor Tree where
  fmap = mapTree
\end{minted}

Applying this same reasoning to the rewrite rule, one might feel enticed to write:

\begin{listing}[H]
\begin{minted}[linenos]{haskell}
{-# Rules
   "fmap/fmap"   forall f g.   fmap f . fmap g    = fmap (f. g)   ;
#-}
\end{minted}
\end{listing}

This is not entirely controversial as the soundness of this rule is verifiable under the Functor laws. However,
GHC has decided against enforcing these laws nor performing transformations that require these laws to hold.
This means that defining custom rules per datatype is required to yield maximum performance. Similar situations
arise with the \mono{Applicative} and \mono{Monad} typeclasses.


\subsection{Tug-of-war}
\label{section:introduction:tug_of_war}

A common manifestation of the cascade effect is the tug-of-war between inlining and the application
of rewrite rules. Continuing with the \mono{Tree} example from the previous section, we find that losing this
tug-of-war is the exact reason that the \textit{mapTree/mapTree} rule never fired. Because the inlining operation
was performed first, the left-hand-side of \textit{mapTree/mapTree} no longer occurs in the program and the rule
is rendered non-active. The final optimised function has regrettably converged to the following form:

\begin{minted}[linenos]{haskell}
addFive :: Tree Int -> Tree Int
addFive = \tree -> mapTree (+1) (mapTree (+4) tree)
\end{minted}

The important point is here is not that rewrite rules are inherently flawed (after all, we could easily
drum up a secondary rule that does fire here), but that one optimisation
may open or close the door to many other optimisations down the road.
From this cascade effect follows that the interaction of a Haskell program
with the optimiser may be quite unstable and consequently sensitive to small changes. 
Changes not only in the source but also in the build environment (a minor release of
the compiler comes to mind). Thus, we cannot trust that our successfully optimised program will remain
optimised in the future. We observe that each program we write may require specific, manual, 
effort to be made more efficient.

\subsection{Non-functional requirements: \mono{inspection-testing}}
\label{section:introduction_inspection_testing}

So if we want our programs to not only be correct but also terminate in reasonable amount of time while
not consuming an overly large chunk of resources, we have to identify an extra set of contraints. 
These constraints do not deal with the functionality of the program but rather its compiled form.
This collection of additional constraints are examples of \textit{Non-functional} requirements. 

To illustrate with a real world example, the very popular \mono{text} library makes the following promise:
`\textit{Most of the functions in this module are subject to fusion, meaning that a pipeline of such functions will usually allocate at most one Text value.}'
\cite{inspection_testing}.
\label{section:introduction:inspection_testing}
Like with \mono{addFive} in \cref{section:introduction:tug_of_war}, 
such promises cannot be checked with traditional tests as they do not concern the functionality of the code. 

And as identified by Breitner \cite{inspection_testing}, the aforementioned promise by the \mono{text} library had in fact
been broken in version \mono{1.2.3.2}, shown by the following counter example:

\begin{minted}[linenos]{haskell}
import qualified Data.Text as T
import qualified Data.Text.Encoding as TE
import Data.ByteString

countChars :: ByteString -> Int
countChars = T.length . T.toUpper . TE.decodeUtf8
\end{minted}

Although \mono{countChars} uses a value of type \mono{Text} during the computations, it does not
need to be actually constructed in the final composite definition. As we learn from the definition:

\begin{minted}{haskell}
data Text = Text ByteString Int Int
\end{minted}

\mono{Text} text is merely a \textit{view} into a \mono{ByteString}
by virtue of an offset and length parameter. This means that length reducing operations can cleverly avoid the
costly task of modifying the underlying \mono{ByteString} and instead just change the offset and length parameters.
Now because UTF-16 contains surrogate pairs, the character length of a \mono{Text} value cannot directly be determined
from the byte-length its \mono{ByteString} and still requires $O(n)$ time. However, this does not justify constructing
a concrete \mono{Text} value as opposed to just using the \mono{ByteString} and the offset and length parameters as
separate bindings.

An analogous situation would be a factory with production line workers that
pack and unpack their intermediate results between every exchange in the assembly line. While they may
conveniently communicate about receiving a `car door, a bolt, and a nut' unified as their `input', they
never actually mean to suggest that they wish to receive them packaged together. So too with our \mono{Text} values,
it is mighty handy to communicate about a package of a \mono{ByteString} and two \mono{Int}s, but we never intent to
actually package them at every turn.

But as mentioned, despite the extensive set of rewrite rules that the \mono{text} library has, the
ideal compilation result was not achieved. In itself this example formed the main motivation to 
develop a method to tests these non-functional requirements.
The result is the \mono{inspection-testing} package \cite{inspection_testing}. 
It provides the machinery necessary to add the following statement
directly to the source file, preventing the same regression from occurring in the future.

\begin{minted}{haskell}
inspect $ 'countChars `hasNoType` ''T.Text
\end{minted}

Despite preemptively saving us from future regressions, \mono{inspection-testing} does not help to
identity and path underlying cause. Consider when the test fails at
some point, and you are tasked with finding the root of the problem. In this very example the final 
optimised definition of \mono{countChar} will have undergone many expanding transformations producing 
over 100 lines to painfully sift through without little more ergonomics than a string search. 

It would be possible to get the output at various intermediate moments to gauge where the 
offending \mono{Text} should have been erased, resulting in an extra compilation artifact per transformation.
Usually this means you have to analyze around 20 different versions of the code. 
This is a tedious and error-prone process, not to mention requires relatively highly skilled programmers
with an in-depth understanding of the GHC and its optimiser. 

Moreover, we risk having to repeat this work in the future when any small number of changes
could, through the cascade effect, trigger this test as failing again.

\section{Introducing \mono{HsComprehension}}

We intend to address the tediousness and skill required for exploring interactions of specific 
pieces of Haskell source code with GHC's optimisation pipeline.

We believe that there is an opportunity to improve the way that Haskell programmers reason about
the performance characteristics of their code while simultaneously appealing to a larger audience
of less experienced programmers.

This belief stems first from the results of the yearly
Haskell survey where in 2021 only 16\% of respondents either agreed or strongly agreed with the
statement `\textit{I can easily reason about the performance of my Haskell programs}' \cite{haskell_survey_2021}.
We are not the only one seeing this statistic as a call to action. As recently as the current year,
Young. J. announced work on a complete Haskell optimisation handbook \cite{optimisation_handbook}.
A preliminary version of the book already shows that a section on analysing optimisation transformations
is planned. Although, it is our opinion that a guiding resource -- while certainly extremely helpful --
is not a substitute for better tooling.

Secondly, through conversations with Haskell programmers at Well-Typed -- who are also major contributors
and maintainers of GHC itself -- we learned that it was common practice to create personal tools that
assisted them in analysing intermediate results of the optimisation transformations during the fairly
frequent task of trying to make performance critical sections of programs more efficient. 
This proves that there is a demand for such tools and that a unified solution does not yet exist.

